%%%%%%%%%%%%%%%%%%%%%%% file template.tex %%%%%%%%%%%%%%%%%%%%%%%%%
%
% This is a general template file for the LaTeX package SVJour3
% for Springer journals.          Springer Heidelberg 2010/09/16
%
% Copy it to a new file with a new name and use it as the basis
% for your article. Delete % signs as needed.
%
% This template includes a few options for different layouts and
% content for various journals. Please consult a previous issue of
% your journal as needed.
%
%%%%%%%%%%%%%%%%%%%%%%%%%%%%%%%%%%%%%%%%%%%%%%%%%%%%%%%%%%%%%%%%%%%


\RequirePackage{fix-cm}
%
%\documentclass{svjour3}                     % onecolumn (standard format)
%\documentclass[smallcondensed]{svjour3}     % onecolumn (ditto)
%
%\documentclass[smallextended]{svjour3}       % onecolumn (second format)
\documentclass[twocolumn]{svjour3}          % twocolumn
%
\smartqed  % flush right qed marks, e.g. at end of proof
%
\usepackage{graphicx}
\usepackage[spanish]{babel}
% \usepackage{mathptmx}      % use Times fonts if available on your TeX system
%
% insert here the call for the packages your document requires
%\usepackage{latexsym}
% etc.
%
% please place your own definitions here and don't use \def but
% \newcommand{}{}
%
% Insert the name of "your journal" with
\journalname{Autonomous Robots}
%
\begin{document}

\title{Jderobot open source middleware for robotic, computer vision and home automation applications
%\thanks{Grants or other notes
%about the article that should go on the front page should be
%placed here. General acknowledgments should be placed at the end of the article.}
}
%\subtitle{Do you have a subtitle?\\ If so, write it here}

%\titlerunning{Short form of title}        % if too long for running head

\author{First Author         \and
        Second Author %etc.
}

%\authorrunning{Short form of author list} % if too long for running head

\institute{F. Author \at
              first address \\
              Tel.: +123-45-678910\\
              Fax: +123-45-678910\\
              \email{fauthor@example.com}           %  \\
%             \emph{Present address:} of F. Author  %  if needed
           \and
           S. Author \at
              second address
}

\date{Received: date / Accepted: date}
% The correct dates will be entered by the editor


\maketitle

\begin{abstract}
Insert your abstract here. Include keywords, PACS and mathematical
subject classification numbers as needed.
\keywords{First keyword \and Second keyword \and More}
% \PACS{PACS code1 \and PACS code2 \and more}
% \subclass{MSC code1 \and MSC code2 \and more}
\end{abstract}

\section{Introduction}
\label{intro}

JOSER, ieee chapter, autonomous robots special issue. Software enginering.


Robot Software:
\begin{itemize}
\item Determina el comportamiento del robot
\item No hay una manera universalmente aceptada de programarlos
\item Lenguajes: ensamblador, C, C++
\item {Heterogeneidad}
\begin{itemize}
\item Dispositivos hardware
\item Encapsular funcionalidad
\end{itemize}
\item Empezar de cero con cada robot
\item Requisitos específicos
\item Sistemas operativos y plataformas
\item Simuladores
\end{itemize}

Robotic software requirements:
\begin{itemize}
\item Vivacidad, agilidad (tiempo real)
\item Multitarea (concurrencia, múltiples fuentes de actividad)
\item Distribuido, comunicaciones
\item Interfaz gráfica, depuración
\item Expandible
\item Hardware heterogéneo
\end{itemize}

¿Por qué software libre en robótica?
\begin{itemize}
\item Independencia de fabricantes
\item Varios modelos de robots
\item Libertad para modificar código fuente
\item Aumentar la calidad del software
\item ``Devolver el favor'': Player, Stage, Gazebo, Opencv, GTK,...
\end{itemize}

We proposed a robotic software platform, named Jderobot (http://jderobot.org).

This paper is organized as follows.

\section{Related works}
\label{sec:relatedworks}

%%david
drivers, software platforms and applications.

ROS, ORCA, Miro, RoboComp, Aria, PlayerStage \cite{Gerkey03,vaughan2007}.

\section{Jderobot platform}
\label{sec:jderobot}

Some history. Keep away the cognitive ideas, focus on software architecture. Nace como implementación de arquitectura cognitiva, tesis doctoral. Evolucionó a una plataforma de programación para aplicaciones robóticas, domóticas y de visión computacional (sensores, actuadores, inteligencia).

\begin{itemize}
\item Orientada a componentes distribuidos y multilenguaje
\item La mantiene un grupo de desarrolladores, es abierta
\item {http://jderobot.org}, manuales, descarga, ejemplos
\end{itemize}
Software libre (licencia GPLv3), paquete debian


Arquitectura cognitiva JDE:
\begin{itemize}
\item Comportamiento = {percepción} y {control}
\item Fragmentación en unidades asíncronas concurrentes (esquemas)
\begin{itemize}
\item[-] de percepción elaboran estímulos
\item[-] de actuación toman decisiones
\end{itemize}
\item La colección de esquemas se organiza en \textit{jerarquía} dinámica (JDE)
\item Sigue el paradigma basado en comportamientos
\end{itemize}

Arquitectura cognitiva: Esquemas.
Un {esquema} es un flujo de ejecución independiente, con un objetivo
\begin{itemize}
\item Funcionamiento continuo
\item Se puede activar y desactivar a voluntad
\item Modulable a través de parámetros
\item Su funcionalidad se usa despertándolo y modulándolo
\item {Perceptivos}: producen estímulos (piezas de información) y los mantienen actualizados. Lecturas sensoriales, transformaciones más elaboradas 
\item {Actuación}: toman decisiones para conseguir o mantener un objetivo, comandos a los actuadores o la activación y modulación de otros 
\end{itemize}














\subsection{Software architecture design}

Some history and lessons learnt: 4.3. 

Threads. GUI specific interface. 

\begin{itemize}
\item Component oriented 
\item Iterative execution
\item Concurrent and optional visualization
\item Processes better than threads
\item ICE for communication
\item Explicit standard interfaces
\item Configuration files 
\end{itemize}

multilanguage, distributed intermachine, Android, efficient

Reuse other open-source libraries: OpenCV, Gearbox, GTK, PCL, OpenGL, PlayerStage, Gazebo, GSL, Cwii.


GUI thread separated from control thread.
Iterative control, iterative perception. Component frequency.
Standard interfaces for communication with other components.

Modelo de programación: Componente
\begin{itemize}
\item Cada esquema se materializa en un {componente} software 
\item El funcionamiento continuo como {ejecución iterativa}, iteraciones periódicas a cierto ritmo para percibir o para actuar
\item Ofrece funcionalidad y usa la de otros a través de {interfaces} explícitos
%el componente es la unidad básica
\item Aplicación $=$ conjunto de componentes concurrentes que interoperan
\item {Multimáquina} y {multilenguaje}: C, C++, python, Java... 
\item Usa el middleware de comunicaciones ICE (zeroC)
\end{itemize}

Ejecución iterativa:
\begin{itemize}
\item Iteración de control o de percepción
\item component\_cycle 
\item \texttt{gettimeofday}
\item \texttt{usleep}
\item Espera condicional
\item Patrón de diseño \textit{Algoritmo iterativo}
\end{itemize}

Interfaces tipados
\begin{itemize}
\item {Comunicación entre componentes}, conexión con otros 
\item Fijan la estructura de los datos que el componente ofrece o necesita
%\item Invocar funciones del interfaz
\item Cada componente puede necesitar y ofrecer uno o varios interfaces 
%espacio de nombres
\item Se definen con \texttt{slice}
\item Generación de código para múltiples lenguajes (C++, Java, Python...)
\item Mecanismos de transmisión de datos eficientes
\begin{itemize}
\item transmisión síncrona o asíncrona
\item publicación/subscripción
\end{itemize}
\end{itemize}


Visualización
\begin{itemize}
\item {Modular} y {opcional}: cada componente tiene (o no) su propio GUI
\item Cada componente incluye el código de su interfaz gráfica
%\item Activable/desactivable a voluntad, cuando interesa
\item Mostrar e interacción con humano 
\item Depuración, útil en desarrollo
\item GTK, XForms, OpenGL
\item Patrón de diseño \textit{Modelo Vista Controlador}
\item Se materializa en una {hebra adicional}
\end{itemize}



Ficheros de configuración:
\begin{itemize}
\item Cada componente puede tener el suyo 
\item Ficheros de configuración ICE
\item Especifican cuál es la fuente de cada interfaz (local o remota)
\item Parámetros específicos
\end{itemize}

\subsection{Interfaces and drivers}


Acceso a los dispositivos hardware
\begin{itemize}
\item Se da a través de {componentes drivers}
\item Acceden localmente al sensor/actuador y ofrecen acceso (local o remoto) a otros a través de interfaces ICE
\item No suelen tener GUI
\item El mismo interfaz de datos lo puede proporcionar varios \textit{drivers}
\begin{itemize}
\item Mismo programa funciona sobre robot real o simulador
\item Independencia de la fuente de imágenes
\end{itemize}
\end{itemize}


Standard interfaces. 
PlayerServer, GazeboServer for Gazebo Simulator \cite{koening2004}, NaoServer, KinectServer, OpenNIServer, giraffeServer, PTU, Hokuyo, Wiimote, CameraServer.

Same interfaces for different robots.
Applications run exactly the same on real robot than on simulator. 

\subsection{Tools and libraries}

\subsubsection{FuzzyLib}: control borroso
\subsubsection{VisionLib}: geometría espacial, analisis de imágenes 
\subsubsection{Progeo}: relacionar las imágenes con 3D, cámaras calibradas 
\subsubsection{Colorspaces}: espacios de color para imágenes

\subsubsection{Control template}
Basic component for reactive controllers
\subsubsection{VisualHFSM}
\subsubsection{Calibrator}
\subsubsection{ColorTuner}
\subsubsection{Recorder and replayer}

When we use real robots, one of the greatest difficulties is repeating experiments with the same conditions, whether we want to compare different algorithms or to test several features of the same algorithm. It's almost impossible to reproduce the same environment conditions, since robot hardware behaves diversely, light conditions change if we use cameras, people or objects move or are located in different places, etc.

To avoid all these matters, we have designed a new tool which records the current devices status and reproduces them whenever we want to. There are two components involved in this tool, the first one is the so-called "recorder", it saves in a file the status of all the devices of the robot, such as odometry, laser measures, image cameras, etc; we can configure which devices we want to track when we executes the components and it works with simulators, real Nao robots and real Pioneer robots.

The second components is the "replayer" component, it reads the file saved by the "recorder" component and provides the same ICE interfaces that would provide the real devices. Thus, when an algorithm gets the current devices status, it can't tell if it's obtaining the current devices measures in real time or pre-recorded data.

This tools has been widely used to perform experiments with real robots, to try different parameters and improve our algorithms.

\subsubsection{Gazebo}

Gazebo.

\section{Research and Teaching}

Nao, pioneer,

\subsection{Android mobileTeleoperator}

Android operating system is increasing its market share every day, both with smartphones and tables. Once you develop an application for this platform, it may be used by students, other researchers or even companies. 

We have created an application called "Mobile Teleoperator" to teleoperate either a Pioneer robot with playerserver or a Nao Robot with BICA architecture. Connection among the mobile device and the robots is made through ICE, the same way we do when we use a standard computer, so we don't need to change anything in Jderobot architecture.

There are three ways to teleoperate the robots:
- Arrows: After pressing a button, it sends the command to the robot and the robot keeps his behavior until the stop button is pressed.
- Joystick: While a button is pressed, it send the command to the robot, but if the button is released it sends a stop command automatically.
- Accelerometer: It uses the mobile accelerometers to command the robot depending on our device orientation. To use this behavior you must keep a "safety" button pressed, this button is used to calibrate the accelerometers when it is pressed the first time and to send a stop command automatically once this button is released.

Mobile teleoperator has been used to create demonstrations and to let people who are not used to control robot to teleoperate robots in a easy way.

\subsection{Navigation algorithms}
Navigation algorithms
\subsection{Selflocalization algorithms}

Self-localization is at the moment one of the most important challenges in robotics. Using robot sensors, such as cameras, laser sensors or ultrasonic sensors, our robot must be able to calculate its own localization inside an environment. Once the robot knows its position, it can adapt its behavior depending on where it is located. 
 
However, robot self-localization has proven to be one the most complex task on mobile robots, since they must face unknown situations, such as occlusions, or being located inside a dynamic environment.

With Jderobot, we have designed and implemented plenty of localization algorithm, since classical localization algorithms such as Monte Carlo particle filters or Kalman filters, to new and improved algorithms such as evolutionary algorithms or MonoSLAM.

\subsubsection{MonoSLAM}

Monocular Simultaneous Localization And Mapping (MonoSLAM) is a type of localization first presented by Andrew J. Davison in 2003 (ref). MonoSLAM is able to construct a point-based map of the environment with a single camera, and localizates the camera inside this environment in real time.

We have developed our own MonoSLAM approach, based on Davison work, obtaining images from real cameras and simulators. We have implemented the point-based approach designed by Andrew Davison and afterwards we have designed our own implementations based on points and lines.

This type of localization is more accurate than classical localization methods and is very useful when robot odometry is not available or is not reliable. On the other hand, it is not able to handle occlusions and needs a faster frame rate.

% For two-column wide figures use
\begin{figure*}
% Use the relevant command to insert your figure file.
% For example, with the graphicx package use
  \includegraphics[width=7cm]{figs/programacion3.jpg}
% figure caption is below the figure
\caption{Please write your figure caption here}
\label{fig:2}       % Give a unique label
\end{figure*}


\section{Conclusions}

Jderobot, open source project
\begin{itemize}
\item Más de 60000 líneas de código
\item Comunidad de usuarios y desarrolladores
%\item Aumentar la calidad del software
\item ¿Dónde conseguirla? ¿Cómo preguntar dudas? 
\item Página web: {http://jderobot.org}
\item Listas de correo: jde-users@gsyc.es y jde-developers@gsyc.es
\item Svn, trac, blog, mediawiki
\item Paquete debian: \texttt{apt-get install jderobot}
\end{itemize}

\begin{acknowledgements}
Funding agencies. Alex, Maikel, Fran.
\end{acknowledgements}

% BibTeX users please use one of
\bibliographystyle{spbasic}      % basic style, author-year citations
%\bibliographystyle{spmpsci}      % mathematics and physical sciences
%\bibliographystyle{spphys}       % APS-like style for physics
\bibliography{bibliografia}   % name your BibTeX data base


\end{document}


